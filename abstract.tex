\abstract{
This dissertation describes methods used for characterizing the sources and effects of environmental noise in Advanced LIGO detectors from the end of the second LIGO-Virgo observing run through the end of the third observing run.
We present the vibrational and magnetic noise budgets for the Hanford and Livingston observatories, results of noise studies focusing on scattered light sources in the Hanford detector and input beam jitter at both detectors, and a system of routine magnetic coupling measurements to track changes throughout an observing run.
We describe the implementation of an automated pipeline for determining the effect of environmental noise signals on gravitational wave events.

Noise plays a critical role in understanding the limitations and areas of potential improvement for analyses searching for gravitational waves.
We discuss the targeted search for gravitational waves associated with gamma-ray bursts during the third observing run.
No gravitational waves were detected coincident with the 86 GRBs analyzed.
New upper limit exclusion distances are set for various signal waveform types, and we report exclusion probabilities for specific models.
We discuss the relevance of transient noise on the performance of unmodeled search pipelines.

This disseration contains previously published co-authored material.
}
