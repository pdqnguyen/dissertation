\appendix

% Changes the table and figure counting to A.1 style
\renewcommand\thefigure{\thechapter.\arabic{figure}}
\renewcommand\thetable{\thechapter.\arabic{table}}
\setcounter{figure}{0}
\setcounter{table}{0}

\chapter{PEMcoupling examples}\label{app:pemcoupling}

This appendix presents examples of files produced by the \pemcoupling package to give readers some guidance on interpreting the outputs.

There are two types of figures produced by \pemcoupling at the single-injection, single-sensor level: a coupling function plot and an estimated ambient plot.
Coupling function plots are produced in the physical, calibrated, sensor units (Figure~\ref{fig:pemcoupling-cf-physical}), as well as in raw counts (Figure~\ref{fig:pemcoupling-cf-raw}).
In the former case the units will look like [m/T] for magnetometers, while in the latter the units are [m/counts], regardless of sensor type.
Estimated ambients are always shown in calibrated units (Figure~\ref{fig:pemcoupling-cf-ambient}).
Table~\ref{tab:pemcoupling-format} summarizes the contents of the output ASCII data output file.
The data will be preceded by a commented line (starting with a \code{\#} character) denoting the comb fundamental frequency if the injection was a comb injection, e.g. \code{\#combfreq:7.10}.

\begin{figure}
  \centering
  % \includegraphics[width=\textwidth]{/path/to/figure}
  \caption{}
  \label{fig:pemcoupling-cf-physical}
\end{figure}

\begin{figure}
  \centering
  % \includegraphics[width=\textwidth]{/path/to/figure}
  \caption{}
  \label{fig:pemcoupling-cf-raw}
\end{figure}

\begin{figure}
  \centering
  % \includegraphics[width=\textwidth]{/path/to/figure}
  \caption{}
  \label{fig:pemcoupling-cf-ambient}
\end{figure}

\begin{table}
	\renewcommand{\arraystretch}{1.5}
	\begin{tabular}{|ll|}
		\hline
		\multicolumn{1}{|l}{\textbf{column}} & \multicolumn{1}{l|}{\textbf{description}}\\ \hline
		\code{frequency}      & bin center frequency {[}Hz{]}\\
		\code{factor}         & coupling factor in {[}m/calibrated sensor unit{]}\\
		\code{factor\_counts} & coupling factor in {[}m/ADC count{]}\\
		\code{flag}           & ``Measured", ``Upper Limit", ``Thresholds not met", or ``No data"\\
		\code{sensINJ}        & sensor amplitude at injection time {[}calibrated sensor unit$/\rthz${]}\\
		\code{sensBG}         & sensor amplitude at background time {[}calibrated sensor unit$/\rthz${]}\\
		\code{darmINJ}        & \ac{GW} channel amplitude at injection time $[\meter/\rthz]$\\
		\code{darmBG}         & \ac{GW} channel amplitude at background time $[\meter/\rthz]$\\ \hline
	\end{tabular}
	\caption{Column descriptions for the single-injection coupling function output of the \pemcoupling package.}
  \label{tab:pemcoupling-format}
\end{table}

Figures for composite coupling functions produced by \pemcoupling follow the same organization: a physical coupling function (with colors representing different injections, Figure~\ref{fig:pemcoupling-ccf-physical}), a raw coupling function (Figure~\ref{fig:pemcoupling-ccf-raw}), and an estimated ambient. (Figure~\ref{fig:pemcoupling-ccf-ambient}).

\begin{figure}
  \centering
  % \includegraphics[width=\textwidth]{/path/to/figure}
  \caption{}
  \label{fig:pemcoupling-ccf-physical}
\end{figure}

\begin{figure}
  \centering
  % \includegraphics[width=\textwidth]{/path/to/figure}
  \caption{}
  \label{fig:pemcoupling-ccf-raw}
\end{figure}

\begin{figure}
  \centering
  % \includegraphics[width=\textwidth]{/path/to/figure}
  \caption{}
  \label{fig:pemcoupling-ccf-ambient}
\end{figure}

Lastly I include examples of the coupling function (Figures~\ref{fig:pemcoupling-sitewide-physical} and~\ref{fig:pemcoupling-sitewide-raw}) and estimated ambient (Figure~\ref{fig:pemcoupling-sitewide-ambient}) plots.

\begin{figure}
  \centering
  % \includegraphics[width=\textwidth]{/path/to/figure}
  \caption{}
  \label{fig:pemcoupling-sitewide-physical}
\end{figure}

\begin{figure}
  \centering
  % \includegraphics[width=\textwidth]{/path/to/figure}
  \caption{}
  \label{fig:pemcoupling-sitewide-raw}
\end{figure}

\begin{figure}
  \centering
  % \includegraphics[width=\textwidth]{/path/to/figure}
  \caption{}
  \label{fig:pemcoupling-sitewide-ambient}
\end{figure}


% Restart counting to B.1, B.2...
\setcounter{figure}{0}
\setcounter{table}{0}

\chapter{PEMcheck examples}\label{app:pemcheck}

This appendix presents screenshots of \code{pemcheck} output pages tested on GW events and environmental noise transients.

\begin{figure}
  \centering
  % \includegraphics[width=\textwidth]{/path/to/figure}
  \caption{}
  \label{fig:pemcheck-GW190510g}
\end{figure}
