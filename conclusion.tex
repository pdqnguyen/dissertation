\chapter{Conclusion}\label{ch:conclusion}

Noise plays a profound role in the field of gravitational wave astronomy.
Understanding how the many sources of noise affect the sensitivity and data quality of the LIGO detectors is crucial to improving our ability to detect astrophysical signals.
Even in the absence of detections, having better sensitivity allows us to make stronger constraints on models of gravitational wave emission.

In this dissertation I have summarized my contributions to the characterization various vibrational and magnetic noise sources.
I have presented methods for measuring coupling functions, and produced noise budgets representing their effects on both of the LIGO detectors.
The vibrational noise budgets have shown that scattering noise is a dominant issue in the detection band of both interferometers.
The magnetic noise budgets have shown that coupling in the detection band can arise and shift in frequency and amplitude due to changes in hardware around the detectors.
Our newly implemented software and hardware infrastructure for making routine magnetic field injections will allow careful monitoring of these changes.

These coupling functions measured for individual sensors have also been essential in developing an automated event validation system for GW candidates.
Excess noise estimates computed from coupling functions have allowed us to quantitatively confirm that environmental signals did not account for any of the gravitational wave detections made by the LIGO collaboration in the third observing run.
The pipeline for O4 has been expanded to incorporate time-frequency information of a GW event and the local background noise of the environmental sensors to produce p-values representing the statistical significance of noise transients overlapping the event.
As the rate of detections continues to increase in the coming years, innovations like these will only become more important, especially if we detect phenomena that have yet to be observed by GW detectors, such as core-collapse supernovae.
