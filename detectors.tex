\chapter{Gravitational wave detectors}\label{ch:detectors}

Detection of gravitational waves requires measuring the transverse stretching and compressing of space. The earliest attempt at this was done through resonant mass detectors, solid, vibrationally isolated cylinders tuned to a particular frequency that could be used to detect the effect of gravitational waves on the length of the cylinders. These proved incapable of reaching the required sensitivity for detecting even the strongest gravitational waves in the frequency band they were designed for ($\sim1$ kHz).

The current era of \ac{GW} detection is dominated by laser interferometers inspired by the simple Michelson interferometer. There are currently three observatories in operation: \ac{LIGO}, consisting of \ac{LHO} in Washington and \ac{LLO} in Louisiana, and the Virgo observatory in Italy. Additional detectors in Japan (Kagra) and India (\ac{LIGO} India) are under construction, and projects for next-generation detectors (Einstein Telescope, Cosmic Explorer, and \ac{LISA}) are on the horizon.

A \ac{GW} interferometer is configured so that the antisymmetric output port (the output not leading back to the laser source) observes no signal due to destructive interference. If one interferometer arm is elongated relative to the other, the phase shift between the signals from both arms results in some constructive interference; thus the amplitude of a gravitational wave passing through the plane of the detector arms is converted to an amplitude in laser light measured at the output port. Sensitivity scales with interferometer arm length, but the incredible length required to detect gravitational waves from even the loudest sources (100s of km in order to observe black hole binary mergers) makes this infeasible. However, the effective arm length can be increased by making the arms Fabry-Perot cavities, so that light travels many ($\sim300$) times back and forth in each arm before leaving through the beamsplitter.

Other methods that have been implemented to dramatically improve the sensitivity current-generation detectors include: increasing laser power or injecting squeezed light to reduce the effects of photon shot noise; implementing a power recycling cavity at the symmetric output that sends the constructively-interfering light leaving in that direction back into the interferometer; implementing a signal recycling cavity at the antisymmetric output to tune the most sensitive frequency band of the detector; larger mirrors to combat radiation pressure on the mirrors due to high laser power~\citep{Creighton_2011}.

The two \ac{LIGO} detectors began their first observing run (O1) on September 12, 2015, and made the first detection of a \ac{GW} from a \ac{BBH} merger on September 14, 2015~\citep{gw150914}, followed by two more \ac{BBH} detections before the end of the run on January 16, 2016~\citep{gwtc1}.
The second observing run (O2) began on November 30, 2016 after a period of detector upgrades and ended on August 25, 2017.
During O2, in addition to several more \ac{BBH} detections, the \ac{LIGO} network (with the addition of Virgo towards the end of the run) observed the first \ac{BNS} merger on August 17, 2017~\citep{gw170817}.
The third observing run (O3), which spanned April 1, 2019 to March 27, 2020, came after another round of major improvements in the performance of the detectors~\citep{Buikema_2020} and the full inclusion of the Virgo detector in the \ac{GW} network.
Throughout this most recent run, ending on Mar 27, 2020, \ac{LIGO} and Virgo observed a total of \XX \ac{GW} events~\citep{gwtc2}.