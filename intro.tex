\chapter{Introduction}

On April 1, 2019, the \ac{LIGO} collaboration ushered forth a new phase of \ac{GW} astronomy as it began its third observing run, which ran in two six-month stages and concluded March 27, 2020.
Together with their European counterpart Virgo, the two \ac{LIGO} detectors would detect a total of 74 new \ac{GW} signals throughout \ac{O3}~\citep{gwtc2, gwtc3}, over three times the detection rate in the first two runs.
In \ac{O1} and \ac{O2}, \ac{LIGO} had already made the first detection of \acp{GW} from the inspiral of \ac{BBH} and \ac{BNS} systems (GW150914 and GW170817, respectively)~\citep{gw150914, gw170817}.
The detections made in \ac{O3} include an additional detection of a \ac{BNS} merger~\citep{gw190425}, the first two detections of \ac{NSBH} mergers~\citep{nsbh_o3}, the first clear detection of an intermediate-mass black hole forming from a \ac{BBH} merger~\citep{gw190521}, and a number of other \ac{BBH} mergers that have expanded and challenged our understanding of black hole populations.

The dramatic increase in detection rate could not have been achieved without the myriad upgrades made to the LIGO and Virgo interferometers themselves~\citep{Buikema_2020}.
These upgrades range from changes in the laser system to replacing core interferometer optics to mitigation of disruptive external signals.
Studying the behavior of the detectors is crucial to finding new ways to improve their sensitivity and stability.
Detector characterization involves deploying a wide array of data analysis tools and experimental tests to understand how noise originating from within and outside of the detectors couples into the \ac{GW} data stream~\citep{Davis_2019, Davis_2021}.

Despite our best efforts, unwanted noise signals still affect the detector in a number of ways.
Short-duration transient signals, called \textit{glitches}, impact analysis pipelines searching for \acp{GW}.
Thus to keep up with the high event detection rates in \ac{O3} many analyses necessary in the validation of \ac{GW} event candidates have been automated, with more sophisticated methods being developed for future observing runs.

This dissertation describes my contributions to \ac{GW} astronomy along multiple avenues. First, Chapter~\ref{ch:gw} introduces gravitational wave emission from a general relativity framework. Chapter~\ref{ch:detectors} describes the anatomy of a ground-based gravitational wave interferometer and the various limitations to its sensitivity. Chapter~\ref{ch:noise-methods} discusses methods used to characterize environmental noise, unwanted signals originating from outside an interferometer. Chapter~\ref{ch:noise-studies} presents the results of those methods during O3, an overview of several noise investigations in which those methods have played a crucial role, and the implementation of an automated algorithm for vetting \ac{GW} detections.

The detection of a \ac{GW} signal coincident with a short \ac{GRB} originating from the \ac{BNS} merger GW170817 was a breakthrough moment for the astronomical community, shedding light on the mysterious properties of \acp{GRB}.
Even as the LIGO and Virgo detectors improve, GW170817 was a fortunate discovery considering its incredibly close proximity (40\,Mpc).
Targeted searches for \acp{GW} associated with \acp{GRB} allow much more sensitive searches for potential joint observations, and are necessary for expanding our ability to make joint detections at greater distances.
Chapter~\ref{ch:grb} describes the connection between \acp{GW} and \acp{GRB}, presents results of searches for joint GW-GRB events during the third observing run and discusses their implications, and remarks on considerations for future analyses.

Finally, Chapter~\ref{ch:conclusion} ends this dissertation with some closing remarks.
